\section{Electrostatic potential}

\begin{framenologo}
  \frametitle{Electrostatic potential}
  \tableofcontents[currentsection]
\end{framenologo}

\subsection{Boundary conditions}

\begin{frame}
  \frametitle{Electrostatic potential}
  \framesubtitle{Boundary conditions}

  Boundary conditions are fulfilled via:
  \begin{itemize}[<+->]
    \item<.-> Electrodes behave bulk

    \item Extended electrode regions helps the above

    \item Corrections to the Hartree potential are necessary (TranSiesta solves Poisson
    equation via Fourier transforms, i.e. periodic)

    \item Always remember that the self-energy \emph{forces} the boundary conditions,
    i.e. long range potential decay may be forced too short

  \end{itemize}

\end{frame}

\begin{frame}
  \frametitle{Electrostatic potential}
  \framesubtitle{Boundary conditions}

  \begin{block}<+->{Hartree potential}

    NEGF calculations are \emph{very} different from periodic calculations because of the
    semi-infinite leads.

  \end{block}

  \begin{block}<+->{Boundary conditions}

    The electrode Hartree potential are the boundary conditions.

    In TranSiesta this is accomplished by fixing the electrostatic potential at one of the
    electrode planes (the one farthest from the device region).

    \begin{itemize}
      \item For $N_\idxE=2$ and aligned semi-infinite directions the potential profile can
      easily be approximated by a linear ramp.

      \item For un-aligned semi-infinite directions (or $N_\idxE\neq2$) the potential
      profile cannot easily be approximated.

      \begin{itemize}[<+->]
        \item By default, TranSiesta implements a \emph{box} solution (bad guess) which
        only defines the boundary conditions on the electrode atoms. I.e. sets the Hartree
        potential equal to the chemical potential in the cell region of the electrode.

        \item A much better approach is to provide an \emph{external} potential profile
        guess, i.e. solve the boundary conditions using external Poisson
        solvers\footnote{This may seem like a more daunting task than it is. The guess is
            the Poisson solution \emph{without} charges, only the boundary conditions.}

      \end{itemize}
    \end{itemize}
    
  \end{block}
  
\end{frame}


\subsection{Multi-electrode electrostatic potential}

\begin{frame}
  \frametitle{Electrostatic potential}
  \framesubtitle{Multi-electrode electrostatic potential}

  \begin{itemize}
    \item 6-electrode system
    \item Default \emph{box} solution for the Poisson equation
    \item External Poisson solution with electrode regions correct chemical potential
  \end{itemize}

  \begin{center}
    \incg[width=.8\linewidth]{ts-poisson-initial}
  \end{center}
  
\end{frame}

