\section{Non-equilibrium Green function}

\begin{framenologo}
  \frametitle{Non-equilibrium Green function}
  \tableofcontents[currentsection]
\end{framenologo}

\subsection{Variables}

\begin{frame}
  \frametitle{Non-equilibrium Green function}
  \framesubtitle{Variables}

  \begin{block}<+->{Important variables in Green function techniques}
    
    \begin{tabular}[c]{>{$}l<{$}|l}
      \idxE & electrode index
      \\
      \HH_\kk & Hamiltonian
      \\
      \SO_\kk & Overlap, for orthogonal basis sets equals $\mathbf I$
      \\
      \DM_\kk & Density matrix
      \\
      \SE_\kk(\E)& Self-energy (not necessarily an electrode!)
      \\
      \Scat_{\idxE,\kk}(\E) & Scattering matrix from $\idxE$
      \\
      \G_\kk(\E) & Green function
      \\
      \Spec_{\idxE,\kk}(\E) & Spectral function originating from $\idxE$
      \\
      \T_{\idxE\mto\idxE'}(\E) & Transmission function from $\idxE$ to $\idxE'$
      \\
      \T_\idxE(\E) & Total transmission function out of $\idxE$
    \end{tabular}

  \end{block}
  
  \begin{block}<+->{Basic equations for Green function techniques}
    \vskip-2ex
\begin{align*}
  \G_\kk(\E)&=\big[(\E+\im\eta)\SO_\kk - \HH_\kk - \sum_\idxE\SE_{\idxE,\kk}(\E-\mu_\idxE)\big]^{-1}
  \\
  \Scat_{\idxE,\kk}(\E) &=i\big(\SE_{\idxE,\kk}(\E-\mu_\idxE)-\SE^\dagger_{\idxE,\kk}(\E-\mu_\idxE)\big)
  \\
  \Spec_{\idxE,\kk}(\E) &=\G_\kk(\E)\Scat_{\idxE,\kk}(\E)\G_\kk^\dagger(\E)
  % \\
  % \DM & =\frac1{2\pi}
  % \iint_\BZ\dEBZ\cd \kk \dd\E\, \sum_\idxE\Spec_{\idxE,\kk}(\E) n_{F,\idxE}(\E)
  % \eikr,
\end{align*}
  \end{block}

\end{frame}


\subsection{Density of states}

\begin{frame}[label=DOS]
  \frametitle{Non-equilibrium Green function}
  \framesubtitle{Density of states}
  
  \begin{itemize}
    \item Density of states over all orbitals
    \begin{align*}
      \mathrm{DOS}(\E) &= \Tr[\DM(\E) \SO]
      \\
      \mathrm{DOS}(\E) &= -\frac1\pi\Im\Tr[\G(\E) \SO]
      \\
      \mathrm{ADOS}(\E) &= \frac1{2\pi}\Re\Tr[\Spec_\idxE(\E) \SO]
      \\
      \mathrm{DOS}(\E) &=\sum_\idxE \mathrm{ADOS}(\E) + \text{bound states}
    \end{align*}

    \vspace{-2ex}

    \item Local density of states on orbital $\nu$
    \begin{align*}
      \mathrm{DOS}_\nu(\E) &= [\DM(\E) \SO]_{\nu,\nu}
      \\
      \mathrm{DOS}_\nu(\E) &= -\frac1\pi\Im[\G(\E) \SO]_{\nu,\nu}
      \\
      \mathrm{ADOS}_\nu(\E) &= \frac1{2\pi}\Re[\Spec_\idxE(\E) \SO]_{\nu,\nu}
      \\
      \mathrm{DOS}_\nu(\E) &=\sum_\idxE \mathrm{ADOS}_\nu(\E) + \text{bound states}_\nu
    \end{align*}

    \vspace{-2ex}

    \item<2->%
    The overlap matrix is extremely important when calculating the density of states!

    \item<3->%
    $\SE$ broadens the DOS similarly to a large $\eta$ value, for states
    coupling to the electrodes
    
  \end{itemize}

  \only<2>{\hfill \hyperlink{integration<3>}{\beamergotobutton{Go to Rules of
              integration}}}

\end{frame}


\subsection{Transmission}

\begin{frame}
  \frametitle{Non-equilibrium Green function}
  \framesubtitle{Transmission}

  \begin{block}<+->{Transmission}
    \begin{itemize}
      \item Transmission between any two electrodes $\T_{\idxE\mto\idxE'}$ 

      \item Total transmission out of any electrode $\T_\idxE$ 

      \item Reflection into electrode $\RE_\idxE$, note $\mathbf 1$ is the \emph{bulk}
      transmission (open channels)

    \end{itemize}
    \vskip -2ex
    \begin{align*}
      \T_{\idxE\mto\idxE'}(\E) &=
      \Tr\big[\Scat_{\idxE'}(\E)\G(\E)\Scat_{\idxE}(\E)\G^\dagger(\E)\big]\quad\text{, for $\idxE\neq\idxE'$}
      \\
      \T_{\idxE}(\E) &\equiv\sum_{\idxE'\neq\idxE}\T_{\idxE\mto\idxE'} (\E)
      \\
      \RE_\idxE(\E) &=\mathbf 1 -\T_\idxE (\E)
      = \mathbf 1 -\Big\{
      \im \Tr\big[(\G(\E)-\G^\dagger(\E))\Scat_\idxE(\E)\big]
      -\Tr[\Scat_\idxE(\E) \G(\E)\Scat_\idxE(\E)\G^\dagger(\E)]
      \Big\}
    \end{align*}

  \end{block}

  \begin{block}<+->{Symmetries}
    When time-reversal symmetry applies we have:
    \begin{equation*}
      \T_{\idxE\mto\idxE',\kk} = \T_{\idxE'\mto\idxE,\kk} = \T_{\idxE\mto\idxE',-\kk}
    \end{equation*}
  \end{block}
  
\end{frame}

\begin{frame}
  \frametitle{Non-equilibrium Green function}
  \framesubtitle{Local currents/bond-currents}

  \begin{block}<+->{Orbital/Bond currents}
    Bond currents are \emph{local} currents flowing between two orbitals/bonds and are
    associated with \emph{out-going} states from electrode $\idxE$
    \begin{equation*}
      \JJ_{\idxE,\nu\mu} = \frac e h \Im\big[
      \Spec_{\idxE,\nu\mu}(\HH_{\mu\nu} - \E\SO_{\mu\nu})
      -
      \Spec_{\idxE,\mu\nu}(\HH_{\nu\mu} - \E\SO_{\nu\mu})\big]
    \end{equation*}

    The sum of bond-currents crossing \emph{any} device cross-section is equal to the
    current between the leads. Without the prefactor $e/h$ this also applies for the
    bond-transmissions\footnote{TBtrans calculates bond-\emph{transmissions}, although
        they are named \emph{currents}.}.

    The bond-currents obey Kirchoff's laws.
    
  \end{block}
  
  \begin{block}<+->{Symmetries}
    \emph{NOTE}, even when time-reversal symmetry is present
    \begin{align*}
      \JJ_{\idxE,\kk,\nu\mu} &= - \JJ_{\idxE,\kk,\mu\nu}
      \\
      \JJ_{\idxE,\kk,\nu\mu} &\neq \JJ_{\idxE,-\kk,\nu\mu}
    \end{align*}
    hence when calculating bond-currents for periodic structures it is imperative to
    average the \emph{entire} Brillouin zone. This may be understood from \emph{momentum}
    considerations. 
  \end{block}
  
\end{frame}


\begin{frame}
  \frametitle{Bond-current in periodic calculations}
  \framesubtitle{Importance of Brillouin-zone averaging}

  \vskip -6em
  \begin{center}
    \begin{itemize}
      \item Square lattice
      \item Periodicity along $x$ direction
      \item Transport along $y$ direction
      \item Bond-currents $k$-averaged
    \end{itemize}
  \end{center}

  \begin{columns}
    
    \column{.5\linewidth}
    \incg[width=.99\linewidth]{trs}
    
    \begin{itemize}
      \item Sampled $k_x\in \big[0;\frac12\big]$
    \end{itemize}

    \column{.5\linewidth}
    \incg[width=.99\linewidth]{no_trs}

    \begin{itemize}
      \item Sampled $k_x\in \big]-\frac12;\frac12\big]$
    \end{itemize}

  \end{columns}

\end{frame}




\begin{frame}
  \frametitle{Non-equilibrium Green function}
  \footnotesize

  \vskip -2ex
  \begin{columns}

    \column{.5\textwidth}
    \begin{block}{Basic equations}
      \vskip-2ex
\begin{align*}
  \G_\kk(\E)&=\big[(\E+\im\eta)\SO_\kk - \HH_\kk - \sum_\idxE\SE_{\idxE,\kk}(\E-\mu_\idxE)\big]^{-1}
  \\
  \Scat_{\idxE,\kk}(\E) &=i\big(\SE_{\idxE,\kk}(\E-\mu_\idxE)-\SE^\dagger_{\idxE,\kk}(\E-\mu_\idxE)\big)
  \\
  \Spec_{\idxE,\kk}(\E) &=\G_\kk(\E)\Scat_{\idxE,\kk}(\E)\G_\kk^\dagger(\E)
  % \\
  % \DM & =\frac1{2\pi}
  % \iint_\BZ\dEBZ\cd \kk \dd\E\, \sum_\idxE\Spec_{\idxE,\kk}(\E) n_{F,\idxE}(\E)
  % \eikr,
\end{align*}
    \end{block}

    \column{.5\textwidth}
    \begin{block}{Local Density of States}
      \vskip-2ex
      \begin{align*}
        \rho_{\nu}(\E) &= -\frac1\pi\Im[\G(\E) \SO]_{\nu} = \sum_\idxE\rho_\nu^\idxE(\E) +
        \text{bound states}
        \\
        \rho_{\nu}^{\idxE}(\E) &= \frac1{2\pi}\Re[\Spec_\idxE(\E) \SO]_{\nu}
      \end{align*}
    \end{block}

  \end{columns}

  \begin{block}{Transmission}
    \vskip-2ex

    \begin{align*}
      \T_{\idxE\mto\idxE'}(\E) &=
      \Tr\big[\Scat_{\idxE'}(\E)\G(\E)\Scat_{\idxE}(\E)\G^\dagger(\E)\big]\quad\text{, for $\idxE\neq\idxE'$}
      \\
      \T_{\idxE}(\E) &\equiv\sum_{\idxE'\neq\idxE}\T_{\idxE\mto\idxE'} (\E)
      \\
      \RE_\idxE(\E) & %=\mathbf 1 -\T_\idxE (\E)
      = \mathbf 1 -\Big\{
      \im \Tr\big[(\G(\E)-\G^\dagger(\E))\Scat_\idxE(\E)\big]
      -\Tr[\Scat_\idxE(\E) \G(\E)\Scat_\idxE(\E)\G^\dagger(\E)]
      \Big\}
      \\
      I_{\idxE\mto\idxE'} &= \frac{e^2}{h}\iint\cd\E\dd\kk\, \T_{\idxE\mto\idxE'}(\E)[n_{F,\idxE}(\E) - n_{F,\idxE'}(\E)].
    \end{align*}

  \end{block}

  \begin{block}{Orbital current}
      
    \begin{equation*}
      \JJ_{\idxE,\nu\mu} = \frac e h \Im\big[
      \Spec_{\idxE,\nu\mu}(\HH_{\mu\nu} - \E\SO_{\mu\nu})
      -
      \Spec_{\idxE,\mu\nu}(\HH_{\nu\mu} - \E\SO_{\nu\mu})\big]
    \end{equation*}

  \end{block}

  \begin{center}
    Quantities depend on $\int_\BZ$, i.e. required $k$ points are typically \emph{very}
    high for correct energy resolved physical quantities.

    Related to the band-structure, a fine $k$-grid is necessary to correctly reproduce the
    band-structure.
  \end{center}
  
\end{frame}


\subsection{NEGF Algorithm in TBtrans}

\begin{frame}[fragile,label=algorithm]
  \frametitle{Non-equilibrium Green function}
  \framesubtitle{Calculation of physical quantities}
  
  \begin{itemize}
    \item Reduce Green function calculation to region of interest ($D$)
    
    \item Algorithms determined from user-requested quantities
    
  \end{itemize}

  \vskip 4em

  \begin{itemize}
    \item Algorithm:
  \end{itemize}
  \begin{center}
    \incg[]{tbt-algorithm}
  \end{center}

  \begin{tikzpicture}[remember picture,overlay]
    \node at ($(current page.center)+(2.5,1.7)$) {%
        \incg[width=.7\linewidth]{inv-block3}%
    };
  \end{tikzpicture}

  \doicite{Papior \etal: \doi{10.1016/j.cpc.2016.09.022}}

  \hfill\hyperlink{electrode<6>}{\beamergotobutton{Return to electrode-setup}}

\end{frame}



%%% Local Variables:
%%% mode: latex
%%% TeX-master: "talk"
%%% End:
