\section{Input options}

\begin{framenologo}
  \frametitle{Input options}
  \tableofcontents[currentsection]
\end{framenologo}

\subsection{Chemical potentials}


\begin{frame}%[fragile]
  \frametitle{Input options}
  \framesubtitle{Chemical potentials}

  \begin{itemize}
    \item Define list of all different chemical potentials

    \item<2-> Define each, different chemical potential

    \item<3-> Remark their physical meaning

  \end{itemize}

  \begin{tikzpicture}[fixed node]
    \def\chem{chem}
    \def\bias{V/2}

    \only<1-5>{
        \begin{scope}[yshift=-2cm]
          \matrix {
              \node[fdf] {\%block TS.ChemPots}; \\
              \node[fdf,ind,lmark] (chem-1) {\chem-1}; \\
              \node[fdf,ind,lmark] (chem-2) {\chem-2}; \\
              \node[fdf,ind] {...}; \\
              \node[fdf] {\%endblock}; \\
          };
        \end{scope}
    }

    \uncover<2->{
    \begin{scope}[xshift=4.1cm]
      \begin{scope}
        \matrix {
            \node[fdf,rmark] (chem-b-1) {\%block TS.ChemPot.\chem-1}; \\
            \node[fdf,ind] (mu-1) {mu \bias}; \\
            \node[fdf,ind] (kT-1) {temp $k_BT_1$}; \\
            \node[fdf,ind] (E-1) {contour.eq.pole $\langle \mathrm{energy}\rangle_1$}; \\
            %\node[fdf,ind] (NP-1) {contour.eq.pole.n $\langle \mathrm{int}\rangle_1$}; \\
            \node[fdf,ind] {contour.eq}; \\
            \node[fdf,iind] {begin}; \\
            \node[fdf,lmark,iiind] (C-chem-1) {C-chem-1}; \\
            \node[fdf,lmark,iiind] (L-chem-1) {L-chem-1}; \\
            \node[fdf,iind] {end}; \\
            \node[fdf] {\%endblock}; \\
        };
      \end{scope}

      \only<-5>{
      \begin{scope}[yshift=-3.8cm]
        \matrix {
            \node[fdf,rmark] (chem-b-2) {\%block TS.ChemPot.\chem-2}; \\
            \node[fdf,ind] (mu-2) {mu -\bias}; \\
            \node[fdf,ind] (kT-2) {temp $k_BT_2$}; \\
            \node[fdf,ind] (E-2) {contour.eq.pole $\langle \mathrm{energy}\rangle_2$}; \\
            \node[fdf,ind] {\dots};\\

            %\node[fdf,ind] (NP-2) {contour.eq.pole.n $\langle \mathrm{int}\rangle_2$}; \\
            \node[fdf] {\%endblock}; \\
        };
      \end{scope}}
    \end{scope}

    \only<-5>{
        \draw[->] (chem-1) -- ++(1.8,0) to[out=0,in=180] (chem-b-1);
        \draw[->] (chem-2) -- ++(1.8,0) to[out=0,in=180] (chem-b-2);
    }

    }

    \begin{scope}[xshift=8.2cm]
      \uncover<3->{%
          \matrix[yshift=2.5cm] {
              \node {$n_{F,1}(E-$};
              \&
              \node[circle,draw=good] (chem-mu-1) {$\mu$};
              \&
              \node {$,$};
              \&
              \node[circle,draw=bad] (chem-kT-1) {$k_BT$};
              \&
              \node {$)$};
              \\
          };
          \draw[->,good] (mu-1) to[out=0,in=-150] (chem-mu-1);
          \draw[->,bad] (kT-1) to[out=0,in=-140] (chem-kT-1);

      }

      \uncover<3-5>{%
          \matrix[yshift=-3cm] {
              \node {$n_{F,2}(E-$};
              \&
              \node[circle,draw=good] (chem-mu-2) {$\mu$};
              \&
              \node {$,$};
              \&
              \node[circle,draw=bad] (chem-kT-2) {$k_BT$};
              \&
              \node {$)$};
              \\
          };
          \draw[<-,good] (chem-mu-2) to[out=-160,in=0] (mu-2);
          \draw[<-,bad] (chem-kT-2) to[out=-150,in=0] (kT-2);

      }
    \end{scope}

    \uncover<4->{%
        \begin{scope}[xshift=8.5cm,yshift=.75cm]
          \draw (-1,0) -- ++(2,0);
          \draw (0,0) node[below] {$\mu_1$} -- ++(0,1);
          \foreach \pole in {0.2,0.4} {
              \fill (0,\pole) circle (2pt);
          }
          \coordinate (E-pole-1) at (0,0.5);
          \foreach \pole in {0.6,0.8,1} {
              \draw (0,\pole) circle (2pt);
          }
        \end{scope}
        \draw[->,thick,shorten >=4pt] (E-1) to[out=0,in=180] (E-pole-1);
        
    \only<-5>{%

        \begin{scope}[xshift=8.4cm,yshift=-5cm]
          \draw (-1,0) -- ++(2,0);
          \draw (0,0) node[below] {$\mu_2$} -- ++(0,1);
          \foreach \pole in {0.2,0.4,0.6} {
              \fill (0,\pole) circle (2pt);
          }
          \coordinate (E-pole-2) at (0,0.7);
          \foreach \pole in {0.8,1} {
              \draw (0,\pole) circle (2pt);
          }
        \end{scope}
        \draw[->,thick,shorten >=4pt] (E-2) to[out=0,in=180] (E-pole-2);
}
    }

    \uncover<5->{
        \def\eta{0.1}%
        \def\radius{3.25}%
        \def\lineS{-1}%
        \def\poles{4}%
        \def\poleSep{.25}%
        % Calculate alpha angle
        \pgfmathparse{\poleSep*(\poles+.5)/\radius}%
        \edef\betaA{\pgfmathresult}%
        \pgfmathparse{atan(\betaA)}%
        \edef\alphaA{\pgfmathresult}%
        \pgfmathparse{asin(\betaA)}%
        \edef\betaA{\pgfmathresult}%
        \begin{scope}[xshift=10cm, yshift=-2cm,scale=.5]
          
          % The axes
          \begin{scope}[draw=gray!80!black,thick,->]
            \draw (-2*\radius+\lineS-.5,0) -- (\radius+1.5,0) node[text=black,below] {$E$};
            \draw (0,0) -- (0,\radius+.5) node[text=black,left] {$\Im$};
          \end{scope}
          \node[below] (mu-1) at (0,0) {$\mu_1$};
          
          % The specific coordinates on the path
          \coordinate (EB) at (-2*\radius+\lineS,\eta);
          \coordinate (C-mid) at ({-\radius+\lineS-sin(\alphaA)*\radius},{cos(\alphaA)*\radius});
          \coordinate (C-end) at (\lineS,{\poleSep*(\poles+.5)});
          \coordinate (L-end) at (\radius,{\poleSep*(\poles+.5)});
          \coordinate (L-end-end) at (\radius+1,{\poleSep*(\poles+.5)});
          \coordinate (real-L-end) at (\radius,\eta);
          \coordinate (real-L-end-end) at (\radius+1,\eta);
          
          \begin{scope}[thick]
            
            % The path (we draw it backwards)
            \draw[->-=.3,very thick,ok] (L-end) -- node[above right] (L) 
            {$\mathcal L$} (C-end);
            \draw[->-=.333,->-=.666,very thick,bad] (C-end) to[out=90+\betaA,in=\alphaA] (C-mid)
            node[above] (C)
            {$\mathcal C$} 
            to[out=180+\alphaA,in=90] (EB);
            \draw[->-=.25,->-=.75] (EB) -- (real-L-end) node[above left] {$\mathcal R$};

            % draw the continued lines
            \draw[densely dotted] (real-L-end) -- (real-L-end-end);
            \draw[densely dotted] (L-end) -- (L-end-end);

          \end{scope}
          
          % Draw the poles
          \foreach \pole in {1,...,14} {
              \ifnum\pole>\poles
              \draw (0,\pole*\poleSep) circle (2pt);
              \else
              \fill (0,\pole*\poleSep) circle (2pt);        
              \fi
          }
          \node[left,anchor=east] at (0,{\poleSep*(\poles/2+.5)}) {$z_\nu$};

          % correct size
          \path[use as bounding box] (-8,-.5) rectangle ++(13,4.5);

          \draw[densely dotted] (real-L-end-end) to[out=0,in=0] (L-end-end);
          \draw[densely dotted,thick] (EB) -- ++(-.5,0);

        \end{scope}

        \draw[->,thick,bad] (C) -- (C-chem-1);
        \draw[->,thick,good] (L) -- (L-chem-1);

    }
    
    \uncover<6->{
        \begin{scope}[yshift=-3.5cm,xshift=.25cm]
          \matrix {
              \node[fdf,rmark] (dC-chem-1) {\%block TS.Contour.C-chem-1}; \\
              \node[fdf,ind] {part circle}; \\
              \node[fdf,iind] {from -40. eV + V/2 to -10 kT + V/2}; \\
              \node[fdf,iind] {points 25}; \\
              \node[fdf,iind] {method g-legendre}; \\
              \node[fdf] {\%endblock}; \\
          };
        \end{scope}

        \begin{scope}[yshift=-3.5cm,xshift=6cm]
          \matrix {
              \node[fdf,rmark] (dL-chem-1){\%block TS.Contour.L-chem-1}; \\
              \node[fdf,ind] {part line}; \\
              \node[fdf,iind] {from prev to inf}; \\
              \node[fdf,iind] {points 12}; \\
              \node[fdf,iind] {method g-fermi}; \\
              \node[fdf] {\%endblock}; \\
          };
        \end{scope}
        
        \draw[->] (C-chem-1) -- (dC-chem-1);
        \draw[->] (L-chem-1) -- (dL-chem-1);

    }

  \end{tikzpicture}
  
\end{frame}

\subsection{Non-equilibrium contours}


\begin{frame}%[fragile]
  \frametitle{Input options}
  \framesubtitle{Non-equilibrium}

  \begin{block}{Non-equilibrium contour}

    \begin{itemize}
      \item The contour spans the entire bias-window
      \item<3-> One may add additional lines to increase precision at certain energies
    \end{itemize}
    
  \end{block}

  \small

  \begin{tikzpicture}[fixed node]
    \begin{scope}
      \matrix {
          \node[fdf] {\%block TS.Contours.nEq}; \\
          \node[fdf,ind,lmark] (lneq) {neq}; \\
          \node[fdf] {\%endblock}; \\
      };
    \end{scope}

    \uncover<2->{
    \begin{scope}[xshift=6cm]
      \matrix {
          \node[fdf,rmark] (rneq) {\%block TS.Contour.nEq.neq}; \\
          \node[fdf,ind] {part line}; \\
          \node[fdf,iind] {from -5 kT -|V|/2 to |V|/2 + 5 kT}; \\
          \node[fdf,iind] {delta 0.01 eV}; \\
          \node[fdf,iind] {method mid-rule}; \\
          \node[fdf] {\%endblock}; \\
      };
    \end{scope}
    \draw[->] (lneq) -- ++(2.5,0) to[out=0,in=180] (rneq);
    }
  \end{tikzpicture}

  \vskip 2em

  \uncover<4->{
      \begin{tikzpicture}[fixed node]
        \begin{scope}
          \matrix {
              \node[fdf] {\%block TS.Contours.nEq}; \\
              \node[fdf,ind,lmark] (lneq-1) {neq-1}; \\
              \node[fdf,ind,lmark] (lneq-2) {neq-2}; \\
              \node[fdf] {\%endblock}; \\
          };
        \end{scope}

        \begin{scope}[xshift=6cm]
          \matrix {
              \node[fdf,rmark] (rneq-1) {\%block TS.Contour.nEq.neq-1}; \\
              \node[fdf,ind] {part line}; \\
              \node[fdf,iind] {from -5 kT -|V|/2 to 0 eV}; \\
              \node[fdf,iind] {delta 0.02 eV}; \\
              \node[fdf,iind] {method mid-rule}; \\
              \node[fdf] {\%endblock}; \\
          };
        \end{scope}
        \draw[->] (lneq-1) -- ++(2.5,0) to[out=0,in=180] (rneq-1);

        \begin{scope}[xshift=6cm, yshift=-2.4cm]
          \matrix {
              \node[fdf,rmark] (rneq-2) {\%block TS.Contour.nEq.neq-2}; \\
              \node[fdf,ind] {part line}; \\
              \node[fdf,iind] {from prev to |V|/2 + 5 kT}; \\
              \node[fdf,iind] {delta 0.005 eV}; \\
              \node[fdf,iind] {method mid-rule}; \\
              \node[fdf] {\%endblock}; \\
          };
        \end{scope}
        \draw[->] (lneq-2) -- ++(2.5,0) to[out=0,in=180] (rneq-2);
      \end{tikzpicture}
  }
  
\end{frame}


%%% Local Variables:
%%% mode: latex
%%% TeX-master: "talk"
%%% End:
