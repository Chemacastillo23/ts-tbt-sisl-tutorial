
\input ../common.tex


\graphicspath{{fig/}{fig-defence/}{paper-fig/}}
\usepackage[export]{adjustbox}

\institute[2017, Nick R. Papior; DTU Nanotech]{\begin{tikzpicture}
      \node[shape=rectangle split,rectangle split parts=2,anchor=base] at (0,0)
      {DTU: sisl, TBtrans and TranSiesta workshop};
    \end{tikzpicture}}

\usetikzlibrary{calc,matrix,intersections}
\tikzset{
    ->-/.style={postaction={decorate},decoration={%
            markings,mark=at position #1 with {\arrow[line width=2pt]{stealth}}%
        }%
    },%
    ->-/.default=.5,%
    -<-/.style={postaction={decorate},decoration={%
            markings,mark=at position #1 with {\arrowreversed[line width=2pt]{stealth}}%
        }%
    },%
    -<-/.default=.5,%
}
\usepgfplotslibrary{groupplots}

\usepackage{algorithm}
\usepackage{algpseudocode}

\date{26. October 2016}
\title{TBtrans/TranSiesta for smarties}
\author{Nick R. Papior}

\usepackage{animate}

\begin{document}


\colorlet{cm1}{green!80!black}
\colorlet{cm2}{red!80!black}
\tikzset{
    s >/.style={shorten >=#1},
    s >/.default=4pt,
    s </.style={shorten <=#1},
    s </.default=4pt,
    s <>/.style={s >=#1,s <=#1},
    s <>/.default=4pt,
    >=latex,
    font=\footnotesize,
    block/.style={
        local bounding box=localbb,
        execute at end scope={
            \node[rounded corners=5pt,
            inner sep=5pt,
            fit=(localbb),draw,#1] {};
        },
    },
    fdf/.style={
        font=\ttfamily\footnotesize,
        anchor=west,
    },
    connect/.style={
        >=latex,
        ->,
        shorten >=4pt,
        shorten <=1pt,
    },
    my mark/.style={
        rectangle,rounded corners=2pt,very thick,
        inner sep=2pt,
    },
    lmark/.style={
        my mark,draw=cm1,%green!80!black,
    },
    rmark/.style={
        my mark,draw=red!80!black,densely dotted,
    },
    iiind/.style={xshift=6ex},
    iind/.style={xshift=4ex},
    ind/.style={xshift=2ex},
    % Matrix opts
    row sep=.4ex,
    every node/.append style={
        inner sep=.25ex,outer sep=.25ex,
        text height=1.25ex,
    },
    ampersand replacement=\&,
}

\begin{frame}
  \titlepage
\end{frame}

\begin{frame}
  \frametitle{Outline}
  \tableofcontents
\end{frame}

\input electrodes.tex
\input reiterate.tex
\input steps.tex
\input contour.tex
\input chemical.tex
\input pivoting.tex
\input summary.tex

\end{document}
