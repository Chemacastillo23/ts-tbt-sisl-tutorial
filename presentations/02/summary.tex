\section{Setup sequence summary}

\begin{frame}
  \frametitle{Setup sequence summary}

  \begin{enumerate}[<+->]

    \item Create geometry
    \begin{itemize}
      % Geometry setup
      \item Determine the electrode size in the semi-infinite direction

      \item Determine required number of \emph{screening} layers between electrode and device
      \item Finalise structure
    \end{itemize}

    \item Create input options, using either method:
    \begin{itemize}
      \item Create input yourself; chemical potential, electrodes and contours

      \item Pre-4.1 users; use \texttt{ts2ts} (converts old input to new input)

      \item \texttt{tselecs.sh}, capable of creating input for $N$ electrodes, requires
      fine-tuning afterwards(!)

    \end{itemize}
    
    \item Run TranSiesta on electrodes with \emph{high} $\mathbf k$-point sampling along
    the semi-infinite direction ($\ge50$)
    for accurate bulk description of the electronic structure

    \item Analyse the pivoting methods using TranSiesta and \texttt{TS.Analyze}

    \item Run TranSiesta on device at $V=0$

    \item \textbf{ENSURE CONVERGENCE}

    \item Step-by-step increase/decrease the bias and copy the \texttt{siesta.TSDE} from
    the nearest lower bias to the folder for restart capability

    \item Run TBtrans, and possibly use the interpolation scheme for precise $IV$ curves
    
  \end{enumerate}

\end{frame}

%%% Local Variables:
%%% mode: latex
%%% TeX-master: "talk"
%%% End:
